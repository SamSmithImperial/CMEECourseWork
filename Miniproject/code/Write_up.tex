\documentclass[11pt]{article}
\usepackage{graphicx} % Required for inserting images
\graphicspath{{../results}}
\usepackage{csvsimple}
\usepackage{booktabs}
\usepackage{subcaption}
\usepackage{setspace}
\usepackage{lineno}


\title{\vspace{+0cm}Does Temperature Influence Model Suitability and Maximum Growth Rates in Microorganisms}

\author{Sam Smith}

\date{}
\onehalfspacing

\begin{document}

  \maketitle
    
  \begin{abstract}
    
  \end{abstract}
  
  \section{Introduction}
  \linenumbers
Microorganisms make up the most diverse and plentiful group of life on Earth. They are essential for ecosystem function and stability \cite{Shoemaker2021}. In particular microbes play a critical role in processes such as the nitrogen cylcle and carbon sequestration that are becoming increasingly significant in light of climate change \cite{Gupta2016}. Microbes have immense ability to multiply rapidly and exponentially whilst resources are abundant. As nutrients become limited, Microorganisms must compete thus compromising their ability to reproduce \cite{NatRevMicro}. This results in them typically showing a sigmoidal pattern of growth whereby different stages of their growth curve infer different biology significance. Microbial growth curves typically consist of a lag phase (tlag), exponential growth phase, stationary phase (K), and often a death phase \cite{Zwietering1990}. The lag phase accounts for the time taken for a population to adjust to its environment before rapid reproduction \cite{BUCHANAN1997313}. Whereas, the stationary phase or carrying capacity is shown as an asymptote on a bacterial growth curve and indicates often compounding factors such as nutrient availability, predation, and crowding preventing further population growth \cite{WACHENHEIM2003157}.\\
     
Models can be developed that effectively predict microbial populations at a given time point. They work under the assumption that microbial population patterns are reproducible under the same environmental conditions \cite{Pla2015}. Modelling microbial population dynamics has positive implications on agriculture and food security as predictions on shelf life and product safety can be made \cite{Zwietering1990}. In addition, predictive models assist in the decision making of large industrial processess such as fermentation \cite{Garcia2021}.Models that are derived from existing theory and recorded observations are known as mechanistic; whilst phenomenological models are developed empirically and provide no explanation of patterns \cite{doi:10.1080/10408398.2011.570463}. Two examples of commonly used non-linear mechanistic models are the the Gompertz model and the Logistic model.\\

The Logistic equation is popular for describing bacterial population growth. The logistic model that was first developed by Pearl and Reed in 1920 was emperical. However, biological inferences are now commonly derived from its parameters \cite{WACHENHEIM2003157} \cite{Pearl1920}. Therefore, it can be said that the logistic model estimates the population at any given time point from the initial population (N0), carrying capacity (K) and maximum growth rate (Rmax) \cite{WACHENHEIM2003157}. Whilst models consisting of fewer parameters are often preferred, the logistic equation does lack parameters that captivate other typical stages of microbial growth curves such as a death or lag phase. The Gompertz equation is another sigmoidal model that takes into account the three parameters in the Logistic model but also includes a lag phase. The Gompertz model has been widely used across a range of applications such as predicting plant, fish or even cancerous tumour growth \cite{Tjrve2017}. Finally, phenomenological cubic linear models are often used for predicting microbial populations. Garcia et al (2021) found that a cubic model accurately represents all aforementioned stages of bacterial population growth when applied to fermentation bacteria \cite{Garcia2021}. However, biological inferences cannot be drawn from phenomoligcal model parameters, highlighting the importance of mechanistic models that provide parameter estimations, as well as predictions.\\

Statistical analysis whereby different models are selected or ranked in terms of performance have begun to gain traction in ecology and evolution \cite{JOHNSON2004101}. It offers an alternative to traditional hypothesis testing techniques as it confronts multiple 'competing' hypotheses simultaneously. It is crucial to identify the best performing models under various conditions, as the prediction of microbial population dynamics relies on selecting the best-performing model that aligns with the conditions of unsampled populations. This report aims to identify the top performing model (Logistic, Gompertz or Cubic) whilst also examining trends in model performance under different temperatures. It is expected that the Gompertz model may drop in performance relative to the Logistic model with increasing temperatures. This is due to the lag phase being significantly reduced by increasing temperatures \cite{ABA2021109108}. Furthermore, I expect no changes in model perfomance for the Cubic model across temperatures as its parameters are not tied to any biological values that are temperature dependent. Temperature is also known to influence the Rmax\cite{Ward1972} \cite{Dey2020}. Rates of substrate uptake by bacteria reduce with lower temperatures thus negatively impacting a bacteria's ability to grow \cite{Nedwell1994}. However as both mechanistic models contain an Rmax parameter this should not vary model performance.\\

Finally, the model averaging application of model selection techniques will be carried out in this study. Robust parameter estimates for K, N0 and Rmax will be calculated using Akaike weights for demonstration and use in potential follow up studies.\\

- write more on how akaike weights have been used in literature
- mention parsimony regarding R2
\section{Methods}

    \subsection{Data Collection}
This study is based on an amalgumation of microbial growth curves from 10 different research papers. It includes populations of different species grown under varying temperatures (0-37) and 18 different media. Observations in the dataset were deemed from the same curve if they shared a temperature, species and citation. This facilitated the subsetting of the data into 285 individual growth curves for model fitting.

    \subsection*{Model Fitting}
Only three canditate models were considered in this study as it is ill-advised to include many models that increases the chance of spurious findings \cite{JOHNSON2004101}. The Logistic model used in this study is the solution to the differential equation defining the classic logistic population growth equation.
    \subsubsection{Logistic Model}
    \begin{equation}
        N_t = \frac{N_0Ke^{r_{max}t}}{K+N_0(e^{r_{max}t} - 1)}
      \end{equation}\\
The Gompertz model used in this study is a modified version by Zwietering et al (1990) \cite{Zwietering1990} where Nmax represents carrying capacity. 
    \subsubsection{Gompertz Model}
    \begin{equation}
        log(N_t) = N_0 + (N_{max}-N_0)e^{-e^{r_{max}exp(1)\frac{t_{lag}-t}{(N_{max}-N_0)log(10)}+1}}
        \end{equation}
Finally, the equation below illustrates the cubic equation and its uninterpretable parameters.
    \subsubsection{Cubic Model}
    \begin{equation}
        log(N_t) = at + bt^2 + ct^3 + d
    \end{equation}\\

Fitting the non linear mechanistic models (Logistic and Gompertz) required reasonable starting values in addition to suitable upper and lower bounds. Sampling was used to vary the starting values around an appropriate mean to increase the likelihood of a non-linear least squares (NLLS) model fit after 100 attempts. After the 100 attempts for each growth curve, only the fit with the highest R2 value was outputted alongside its respective AICc, BIC, and Akaike weight score. All three models were managed to be fit to 202 out of 285 growth curves. For consistency, when working out the statistical metrics for the logistic model, the residuals were log transformed to facilitate comparison between the other models. In addition, log transformed population data is prefered as population is a multiplicative process therefore residuals naturally increase with time in a linear scale. This will increase the normality of the model residuals which in turn increases the predictive power of the models \cite{Freckleton2002}.


    \subsection{Model Comparison and Weighted Averages}
Johnson and Omland outlined model selection metrics and described the differences between them. Firstly, R2 is used that simply suggests the proportion of the variance in the data that is explained by a given model. However, R squared is not commonly used and is described as 'naive'as it does not consider model complexity and therefore does not penalise overfitted models \cite{JOHNSON2004101}. Secondly, Akaike information criterion (AIC) is another measure of model performance that counts for goodness of fit and model complexity. Furthermore, AICc includes bias correction when the model is applied to small sample sizes. Finally bayesian information criterion (BIC), similar to AICc, considers fit, complexity and sample size. However, AIC is often favoured as it is based on Kullback-Leibler information theory \cite{JOHNSON2004101} - although many statisticians argue that BIC is preferable as it is less tolerant of overcomplex models. When comparing model AIC and BIC values, models with the smallest AIC and BIC by at least 2, are considered the better performing model \cite{JOHNSON2004101}. Model comparison occurs between models fit onto the same data and an overall model winner is defined as the model that performed best the most number of times. in addition, model performances were compared under all different temperature values to identify trends in model performance.\\

Omland and Johnson further explain the use of Akaike weights that consider the relative probabilities of a model being the best performer. In this study, for a model to be selected as the best performer an akaike weighted score of over 0.9 is necessary \cite{DASH2023103140}. This is an arbitrary threshold that suggest that a model can only be deemed the best if there is more than 90 percent chance that that is the case. These probabilities can consequently be used for robust parameter estimations. In this study, Akaike weighted averages were made for comparison between all three models. However, for parameter estimations akaike weights were recalculated for just Gompertz and Logistic as the Cubic model does not contain interpretable parameters associated with stages of bacterial growth.

    \subsection{Computational Tools}

\section{Results}
All 202 growth curves with fitted models were plotted with both log transformed and linear population values. For example subset 225 is a good example for illustrating the differences between plot types:\\

\begin{figure}[htbp]
    \centering
    \begin{subfigure}{0.5\textwidth}
      \includegraphics[width=\linewidth]{LinearMod.png}
      \caption{Population against Time}
    \end{subfigure}%
    \begin{subfigure}{0.5\textwidth}
      \includegraphics[width=\linewidth]{LogMod.png}
      \caption{Log Population against Time}
    \end{subfigure}
    \caption{Graphs showing the relationship between log and linear microbial populations against time for the subset 225.}
\end{figure}

In both plots it is clear that the Gompertz and Cubic models were best fit to the data. This would suggest their R2 values would be stronger however, as the logistic model contains fewer parameters, AIC and BIC values may tell a different story. Furthermore, both plots illustrate the Logistic models struggle to fit curves with a time lag phase. The Gompertz and Cubic manage to captivate the lag phase, growth phase and stationary phase, with the cubic even predicting a death phase shown by a slight decline in predicted population.\\

The number of times model performed best for each criteria is given in this table below: 
\begin{center}
\csvautobooktabular{../results/bestmodeltable.csv}
\end{center}
The table shows that Gompertz was the best performing model the most number of times for each criteria. The Gompertz model's quality of fit clearly outweighed its penalty for having an extra parameter than the Logistic model for AICc and BIC. Furthermore, the Cubic and Gompertz models clearly outperformed the Logistic according to Rsqrd, aligning with expectations due to their greater number of parameters. Despite this, the number of Akaike weights best fits is fairly similar across all three models. The Gompertz model had a probability of being the best model of more than 0.9 only 59 times out of the 202 growth curves. It is important to note that the columns may not add up to 202 due to 'no clear winner' outcomes.\\

The propotions of best fits for each model were calculated and plotted for each unique temperature value in the dataset. 
  \begin{center}
    \includegraphics[scale=0.5]{proportionplot.png} 
  \end{center}
Observing the plot there is no clear trend between performance for any model and temperature. Whilst there are temperatures which certain models perform best at, for example cubic at 16 degrees, drawing definitive trends is challenging due to small sample sizes for each temperature.

\section{Discussion}

The trade off between parsimony and complexity is an important dilemma to address in this case. Parsimonous theories advocate for simplicity of explanation \cite{Coelho2019} thus the most straightforward model that adequetly explains the data is best. Simplicity is often favoured as it is more interpretable and overfitting is less likely. In contrast, complex models with more parameters than its competitors (such as Gompertz) explain intricacies in the data, thus providing a better fit. Therefore, metrics that quantify the balance of parsimony and complexity, AIC and BIC, were used in this study to identify the best performing models.\\

In this study we identified the Gompertz model to be the best predictor of microbial population growth - implying that the additional parameter compared to the Logistic model provide sufficient explanation of the data to justify its complexity. Moreover, there was no evidence from figure X to suggest that the efficacy all models changed with temperature. Therefore, this report encourages the use of the Gompertz model for predicting microbial populations across a range of temperatures. The results do not align with the expectation that the Logistic model would outperform the Gompertz model at higher temperatures. Reasons for this may be due to the limitations of this study such as the other confounding factors at play. Firstly, the length and extent of different stages of microbial growth will depend on its environmental conditions and the environmental conditions that it is best adapted for \cite{Dey2020}. For example, lag phase increases for microbes placed in novel environments to allow for time taken to adapt to new conditions \cite{Rolfe2012}. This favours the Gompertz model performance where a tlag parameter is present. Secondly, growth mediums are known to affect metabolic rates in bacteria \cite{KIM201764}. Kim and Kim (2017) found that microbes grown in nutrient rich media had increased K and Rmax and decreased tlags than those grown in nutrient poor media. This indicates that the Logistic equation is more suitable for nutrient rich media. Thirdly, different species of microorganisms exhibit different patterns of growth. For example, Tetraselmis tetrahele, a species of phytoplankton included in the study is likely to have a lag phase synonymous with most algal species \cite{ebaac5c3-39dd-33ef-a890-3745d155862c}. This increases the feasability of the Gompertz and Cubic models. To address the limitations of this study, more specified experiments that record enough data for each unique set of conditions would provide a more nuanced understanding of model performance.\\


The limited sized data set also reduces the significance of the findings of the study. Whilst the Gompertz model performed best for all criteria, no statistical analyses were conducted to address whether or not Gompertz would likely be the best performing model in another set of growth curves. Furthermore, the Akaike weight scores illustrate no overwhelming winner and therefore suggest model selection should be more specific to the conditions that a microbe is grown under. For example, when predicting the parameters or population growth of bacteria grown under the same conditions it is adapted to then the Logistic model should be selected for reasons previously mentioned. On the other hand, when handling microbes grown under nutrient poor environment then a model with a lag phase parameter is more suitable. This study did not contain a plentiful enough data set to select the best models for each unique set of environmental conditions. However, to combat this, the Akaike weighted parameter estimations could be used to predict microbial population growth - particularly when there is no clearly best performing model. Model averaging using Akaike weights is mostly useful when multiple models exhibit roughly equal AIC values \cite{JOHNSON2004101}, so equally performing models are both considered when calculating parameters. This study generated Akaike estimated parameter values regardless of whether or not there was a clear best performer (Aw > 0.9). However, a potential improvement to the estimations, as suggested by Johnson and Omland (2004), would be to only use model averaging when the best performing models Akaike weight is less than 0.9. And otherwise just use given parameter of the best performing model. This is to prevent the influence of poor performing models on parameter estimates.

\section{Conclusion}
In conclusion the results of this study provides explanation for why the Gompertz model should be used to gain a general idea of microbial population growth. However, it is likely that selecting models that perform best for particular conditions, or using akaike weights is more reliable method for predicting growth curves and parameters.



 
  \bibliographystyle{apalike}
  
  \bibliography{Bibliography}
\end{document}